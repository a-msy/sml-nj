\documentclass{jarticle}
\begin{document}

\title{非手続き型言語7回目課題}
\author{09430509 今田将也}
\maketitle

\section{1.ラムダ計算とは何か}

ラムダ計算は,アロンゾチャーチとスティーブンコールクリーネによって1930年代に提唱された
ものである.計算の実行を関数への引数の評価と適用としてモデル化・抽象化した計算体型のことで
関数型言語の基礎としてラムダ計算がある.

このラムダ計算には,ラムダ記法というものが使われる.このラムダ記法は関
数と関数の計算結果をきちんと区別したいという理由でできた.
関数を表しているのかそれとも関数に値を入れた時の計算結果を表している
のかを明確に区別しようというのがラムダ記法である.ラムダ記法ではfを関数
そのものでf(x)はその関数にxを適用した時の値とする.そして,関数fにお
いて引数を表現するのにラムダの記号を使う.

\begin{equation}
  f(x) = ax^2+bx+c
\end{equation}
上記の式だとfという関数を表しているのかそれとも関数fに値xを入れたときの
計算結果を表しているのかは文脈依存である.
これを区別しようというのがラムダ記法であり以下の様に書く

\begin{equation}
  f=λx.ax^2+bx+c
\end{equation}

値f(x)は以下のように書く
\begin{equation}
  (λx.ax^2+bx+c)x = ax^2+bx+c
\end{equation}

\section{2.ラムダ式の定義を述べよ}

型なしラムダ計算で用いる式

定義1 λ式の定義
\begin{enumerate}
\item 変数$x_0,x_1,...$はλ式
\item Mがλ式でxが変数のとき$(λx.M)$はラムダ式.(ラムダ抽象)
\item MとNがλ式のとき$(MN)$はλ式
\end{enumerate}

例として変数xに関数fを2回適用して得られるλ式は以下のように書く.
\begin{equation}
  (λx.(f(fx)))
\end{equation}

また,定義1の3つ目の規則はMとNがどんなラムダ式であっても適用できるということ
で高階関数をそのまま表現できている.
\end{document}
